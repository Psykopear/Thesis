\chapter{Problemi di direction-finding}
\fancyhead[RO]{\bfseries Problemi di direction-finding}
\section{Introduzione dei problemi di direction finding}

L'obiettivo delle tecniche di direction finding (DF) è quelllo di determinare la direzione di provenienza di un segnale trasmesso da un'emittente radio, misurando e valutando alcuni parametri dei campi elettromagnetici. Per determinare la direzione è in generale sufficiente calcolare l'azimut\footnote{L'azimut è la distanza angolare compresa tra la direzione del Nord e la direzione in cui cade la perpendicolare di un punto all'orizzonte, calcolata muovendosi in senso orario.}. A questa, solitamente, va sommato il calcolo dell'altitudine, qualora il DF venga installato su una piattaforma volante. Solo nel caso di una propagazione delle onde priva di disturbi la direzione del segnale emesso coincide con quella del segnale ricevuto. L'apparecchio che esegue il direction finding, prende in considerazione le onde ricevute, e le analizza cercando di approsimare al meglio la direzione di arrivo del segnale.

In questo capitolo vengono introdotte le tecniche utilizzate in diversi campi per risolvere questo tipo di problea. In particolare viene spiegato il sistema basato sulla potenza del segnale ricevuto, che è il metodo utilizzato per questo lavoro. 

\subsection{Utilizzi delle tecniche di direction finding}
Sebbene le tecniche di DF legate alla navigazione non siano molto diffuse, data la presenza dei sistemi di navigazione satellitare, la loro richiesta aumenta nei campi dove è importante determinare la posizione di antenne riceventi e trasmittenti. Alcuni casi d'uso riguardano i servizi di sicurezza in cui è richiesto di rintracciare l'origine di comunicazioni radio emesse da organizazioni criminali; l'intelligence militare, per rivelare possibili attività nemiche, o svelare posizioni o ordini impartiti via radio; in ambiti di ricerca quali la radio astronomia o il sensing remoto di dati raccolti. 

Un altro fattore che rende importanti queste tecniche, è il costante aumento delle comunicazioni wireless. Il direction finding è quindi un primo passo fondamentale per la localizzazione radio, specialmente quando il contenuto di queste trasmissioni non può essere letto. La localizzazione delle emittenti è di solito un processo suddiviso in diverse fasi: 
\begin{itemize}
\item Emittenti posizionate in un largo spazio geografico che possono approsimare la posizione con una precisione di pochi chilometri attraverso la triangolazione radio;
\item La posizione dell'emittente può essere raffinata utilizzando riceventi collocate nei veicoli; 
\item Sistemi di DF portatili che permettono di ridurre la ricerca nel raggio di un centinaio di metri, ad esempio all'interno di un edificio.
\end{itemize}


\subsection{Classificazione dei sistemi di direction finding}

In letteratura è presente un vasto assortimento di algoritmi e sistemi per la soluzione di problemi di DF.
I sistemi di "radio direction finding" utilizzano una seria di antenne e una o più riceventi o per stimare un angolo di rilevamento o per determinare le coordinate geografiche di un segnale ricevuto (SOI\footnote{SOI: Signal Of Interest}). La funzione primaria di un sistema di DF è quella di calcolare la direzione di arrivo (DOA\footnote{DOA: Direction Of Arrival}). 
Gli algoritmi di DF sono di solito divisi in due categorie: sistemi di DF ad n canali, che utilizzano un canale della ricevente per ogni antenna; sistemi di DF a canale singolo, che utilizzano una singola antenna, una serie di trasmettitrici e un sistema che permetta di passare da un segnale all'altro, o che combini i segnali in modo da poterli presentare alla ricevente come un segnale singolo. I sistemi a canale singolo hanno alcuni vantaggi rispetto a quelli ad n canali, tra cui la portabilità, a discapito della potenza di calcolo o della robustezza in condizioni avverse. La sfida rappresentata dallo sviluppo di tecniche di DF a canale singolo e ciò che ne fa un interessante campo di studi è quello di ricercare un sistema che permetta di offrire prestazioni tipiche dei più complessi sistemi ad n canali, contribuendo anche allo sviluppo di questi ultimi.

Altre forme di classificazione degli algoritmi di DF prendono come parametro il modo in cui il segnale viene trattato. Gli approcci possono essere, dunque, basati sull'ampiezza del segnale (RSS\footnote{RSS: Received Signal Strenght}), sulla fase	dell'onda ricevuta, o una combinazione delle due. I sistemi basati su RSS comparano l'ampiezza ricevuta dai vari elementi nella serie di antenne per localizzare un punto nel piano vicino alle antenne dalle quali ha origine il segnale. I sistemi basati sulla fase del segnale determinano le informazioni necessarie a stabilire la DOA dalla fase assoluta, o dalla differenza di fase, dei segnali ricevuti dalle antenne. I sistemi che utilizzano entrambi i sistemi sono più complessi, ma in generale hanno prestazioni migliori.

\subsubsection{Sistemi a Tempo Di Arrivo (TOA)}

Il tempo di arrivo di un segnale che viagga da un nodo all'altro può essere utilizzato per stimare la distanza fra i due nodi. Se i due nodi sono sincronizzati, la ricevente può determinare il ToA del segnale. La sincronizzazione fra i due nodi è un punto fondamentale in questo tipo di tecnica e da questa, in genere, dipendono le prestazioni del sistema nel complesso. Per effettuare la stima della posizione sono necessari almeno 3 nodi.

\subsubsection{Sistemi a Direzione di Arrivo (DOA)}

I sistemi basati sulla direzione di arrivo utilizzano di solito stazioni ad antenna multipla. Questi sistemi sono interessanti in quanto, a differenza di altri, non hanno bisogno di una sincronizzazione fra le stazioni, e sono sufficienti due stazioni base per permettere	di effettuare la localizzazione. Ci sono molte soluzioni basate su questo approccio sviluppate negli ultimi quranta anni. Il MUSIC\footnote{Acronimo di MUltiple SIgnal Classification, si tratta di una tecnica per la stima di frequenza e per la localizzazione della direzione di provenienza di un segnale (DOA).}, l'ESPIRT\footnote{Acronimo per Estimation of Signal Parameters via Rotational Invariant Techniques (ESPRIT) è una tecnica per determinare dei parametri di un insieme di sinusoidi immerse in del rumore di fondo.} e le rispettive varianti, sono le più utilizzate in tempi recenti. Questi, pur essendo subottimali, forniscono delle ottime prestazioni e sono di solito considerati algoritmi di riferimento in questo ambito.

\subsubsection{Sistemi Received Signal Strenght (RSS)}
Queste tecniche si basano sull'idea che la potenza del segnale varia con la distanza. In questo modo la RSS inviata alla ricevente dà informazioni riguardo la distanza di questa dalla sorgente. In ogni caso è necessario stabilire una scala che permetta di convertire la differenza di potenza del segnale ricevuto (PL\footnote{PL: Path Loss}) in una distanza. Un modello comune per la PL chiamato "log-normal shadowing PL model" è dato da:
 $$ PL\left(d\right) = PL\left(d_0\right) + 10n_plog_{10} \left( \frac{d}{d_0} \right) + v $$
dove:
\begin{itemize}
\item $ PL(d) $ è il PL in dB alla distanza d; 
\item $ PL(d_0) $ è il PL in dB ad una distanza relativamente breve $ d_0 < d $ ($ d_{0} $ di solito è un metro); 
\item $n_p$ è quello che è chiamato "path loss exponent". I valori di $n_p$ più comunemente utilizzati possono andare da 2 in campo aperto, o 4-6 in caso di utilizzo in un percorso ostruito da muri.
\item $v$ è una variabile gaussiana casuale che rappresenta l'effetto del "log normal shadowing". Solitamente questa variabile $v$ è considerata zero. 
\end{itemize}
Questo modello può essere utilizzato sia per utilizzi interni che all'aria aperta. 

