\section{Sorgente algoritmo geometrico}
\begin{verbatim}
def search_far_calibration(drone):

    distances = drone.distances
    STEP = len(distances)

    # Calibrazione non effettuata

    if STEP <= 2:

        direction = d["E"] if STEP == 1 else d["N"] if STEP == 2 else drone.last_direction
        return void_directions(direction, drone)

    # Calibrazione effettuata, inizio a muovermi

    elif STEP == 3:

        # Le tre misurazioni della "triangolazione" per capire
        # il quadrante in cui si trova il punto d'arrivo

        sud_ovest, sud_est, nord_est = distances[:3]
        direction = "S" if sud_est < nord_est else "N" if sud_est > nord_est else ""
        direction += "E" if sud_ovest > sud_est else "O" if sud_ovest < sud_est else ""
        drone.last_direction = d[direction]

    return go_far(drone) if distances[-1] > 1.0 else search_close(drone)


def go_far(drone):

    if len(drone.distances) > 3 and (drone.distances[-2] - drone.distances[-1]) < 1:

        if drone.last_modifier == 0:

            drone.last_modifier = 1 if random.random() < 0.5 else -1

        if not drone.flipflop:

            drone.flipflop, drone.last_direction = (True, (drone.last_direction + drone.last_modifier) % 8)

        else:

            mod = 3 if drone.distances[-1] > drone.distances[-2] else 1
            drone.flipflop, drone.last_direction = (False, (drone.last_direction + mod) % 8)

    return void_directions(drone.last_direction, drone)


def search_close(drone):

    direction = (0 if drone.last_direction == 1 or drone.last_direction == 7 else
                4 if drone.last_direction == 3 or drone.last_direction == 5 else drone.last_direction)

    return void_directions(direction, drone)


def change_strategy(drone):

    drone.distances = []
    x, y = drone.actual_position
    close_distances = []

    # Controllo in quale dei punti adiacenti sono passato meno volte

    for x_index in (x - 1, x, x + 1):
        for y_index in (y - 1, y, y + 1):

            # Se il punto e' accessibile, e non e' il punto stesso in cui sono partito
            # viene aggiunto all'array

            if x_index >= 0 and x_index < len(drone.graph[0]) and y_index >= 0 and y_index < len(drone.graph[0]):
                if x_index != x or y_index != y:

                    # Questo array conterra' tutti i punti adiacenti ed accessibili

                    try:
                        close_distances.append([drone.kb[(x_index, y_index)][1], x_index, y_index])
                    except:
                        close_distances.append([0, x_index, y_index])

    # Vado verso il primo dei punti in cui sono passato meno volte

    return void_directions(get_direction(min(close_distances)[1] - x, min(close_distances)[2] - y), drone)
\end{verbatim}
	
\section{Sorgente generazione grafo dinamico}
\begin{verbatim}
class Graph(object):

    def __init__(self, x, y):

        self.graph = {}
        self.graph[(x, y)] = (0, 1, 0)
        self.counter = 1

        # Set this to the same amount of Drone.fuel to make
        # the algorithm behave like if there is no optimization
        # If it's less, it will delete nodes when the graph becomes
        # big, saving memory, with a variable increase of the cost
        # of the search algorithm. 20 seemed to be a good choice for
        # this parameter, for matrixes between 5 and 20 of size
        # (obviously not deleting nodes is always better for calculations)

        self.graph_max_length = 2000

    def __getitem__(self, item):

        return self.graph[item]

    def add_node_coord(self, coord):

        self.counter += 1

        if len(self.graph) > self.graph_max_length:

            for node in self.graph:

                if self.graph[node][2] == (self.counter - self.graph_max_length):

                    self.graph.pop(node, None)
                    break

        new_node = Node(coord, 1, self.counter)

        if not new_node.k in self.graph:

            self.graph[new_node.k] = new_node.v

        else:

            old_node = self.graph[coord]
            self.graph[coord] = (old_node[0], old_node[1] + 1, self.counter)

        return new_node.k

    def change_weight(self, coord, w):

        self.graph[coord] = (w, self.graph[coord][1], self.graph[coord][2])

    def print_graph(self):

        for node in self.graph:

            print node, ": ", self.graph[node]

    def goto(self, coord, way):

        new_coord = sum_coord(coord, d[way])
        return self.graph[new_coord] if new_coord in self.graph else -1


class Node(object):

    def __init__(self, k, weight, counter):
    
        self.k = k
        self.v = (weight, 1, counter)

\end{verbatim}
\section{Screenshot iterazioni}
	