\section{Definizione del software di benchmark}
Per valutare la qualità dell'algoritmo sviluppato si è rivelato utile sviluppare un software in grado di raccogliere informazioni sull'efficenza dell'algoritmo stesso in tutte le possibili combinazioni di punto di arrivo/punto di partenza in mondi di dimensioni fissate, osservandone i casi peggiori, la media aritmetica dei casi, lo scarto quadratico medio, e i risultati ottenuti più frequentemente. Il software permette di testare qualsiasi algoritmo in grado di interfacciarsi con il sistema di gioco. Non essendo presenti in letteratura algoritmi che risolvessero lo stesso problema qui trattato, ho ritenuto opportuno fare un confronto con un algoritmo greedy ottimo, che ha però bisogno di informazione completa sugli stati di ogni cella del mondo (anche senza aver fatto il movimento). Il termine di paragone è quindi il rapporto tra i valori raccolti attraverso l'esecuzione dell'algoritmo ottimo e quelli ottenuti con l'esecuzione dell'algoritmo descritto, più questo si avvicina ad 1, più le prestazioni dell'algoritmo sviluppato si avvicinano a quelle dell'algoritmo ottimo. Chiaramente il rapporto non sarà mai pari ad 1 a causa della differenza delle informazioni riguardo il mondo che i due agenti hanno, quindi la valutazione ha un carattere simbolico utilizzata in questo modo. Può però diventare interessante nel caso in cui vengano sviluppati altri algoritmi atti a risolvere la stessa problematica, per un confronto alla pari.
	
\section{Parametri che influenzano la qualità del risultato}
	perchè anche loro valgono
	
\section{Risultati}
	era il caso
