\section{Introduzione dei problemi di direction-finding}

In letteratura è presente un vasto assortimento di algoritmi e sistemi per la soluzione di problemi di DF. Introdurrò qui il problema,
senza entrare nel dettaglio delle singole soluzioni proposte, per poi presentare un approccio differente per la risoluzione dello stesso.

Dato un agente che riceve un segnale da più antenne trasmettitrici, il problema del DF è quello di stabilire la direzione di arrivo,
ovvero la direzione da cui il segnale stesso è generato (nel caso di antenne multiple, una posizione in qualche modo vicina a queste).

Gli algoritmi di DF sono di solito classificati dal numero di riceventi utilizzate. Quelli che utilizzano due o più riceventi, e quelli
a ricevente singola, mobile. Possono essere classificati anche a seconda del metodo di trattamento del segnale, ovvero approcci basati
sull'ampiezza del segnale, sulla sua fase, o su entrambe. 

I sistemi basati sull'ampiezza comparano l'ampiezza del segnale ricevuto dalle varie antenne per calcolare la posizione nel piano da cui
il segnale è generato.

Quelli basati sulla fase determinano la direzione di arrivo traendo informazione dalla fase assoluta, o dalla differenza delle fasi, dei 
segnali ricevuti dalle riceventi.

\section{Risoluzione del problema con riceventi multiple}
Testo ancora

\section{Risoluzione del problema con una ricevente mobile}
Testo a caso