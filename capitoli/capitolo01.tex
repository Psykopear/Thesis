\section{Introduzione dei problemi di direction-finding}

In letteratura è presente un vasto assortimento di algoritmi e sistemi per la soluzione di problemi di DF. Introdurrò qui il problema,
senza entrare nel dettaglio delle singole soluzioni proposte, per poi presentare un approccio differente per la risoluzione dello stesso.

Dato un agente che riceve un segnale da più antenne trasmettitrici, il problema del DF è quello di stabilire la direzione di arrivo,
ovvero la direzione da cui il segnale stesso è generato (nel caso di antenne multiple, una posizione in qualche modo vicina a queste).

Gli algoritmi di DF sono di solito classificati dal numero di riceventi utilizzate. Quelli che utilizzano due o più riceventi, e quelli
a ricevente singola, mobile. Possono essere classificati anche a seconda del metodo di trattamento del segnale, ovvero approcci basati
sull'ampiezza del segnale, sulla sua fase, o su entrambe. 

I sistemi basati sull'ampiezza comparano l'ampiezza del segnale ricevuto dalle varie antenne per calcolare la posizione nel piano da cui
il segnale è generato.

Quelli basati sulla fase determinano la direzione di arrivo traendo informazione dalla fase assoluta, o dalla differenza delle fasi dei 
segnali misurati dalle riceventi.

\section{Utilizzi delle tecniche di Direction Finding}
Queste tecniche sono utilizzate in diversi ambiti. Il Radio Direction Finding, o RDF, è stato in passato il principale aiuto alla navigazione marina, ma al giorno d'oggi questo è stato sostituito dall'utilizzo del GPS.
Durante la seconda guerrda mondiale tali tecniche erano utilizzate per la localizzazione di trasmittenti segreti ed ostili all'interno del suolo inglese da parte del  Radio Security Service, e questo tipo di utilizzo continuò anche nel periodo della guerra fredda, su tutto il suolo europeo.
Ci sono molti tipi di trasmettitori radio utilizzati in caso di emergenze, che sono molto utilizzati sull'aviazione civile, che attraverso l'utilizzo di queste tecniche permettono di rilevare la posizione di origine del segnale.
Gli stessi apparecchi di ricerca in valanga (ARVA) utilizzano tali tecniche, permettendo a chi non è stato sepolto da una valanga di trovare la posizione di chi si ritrova bloccato.
Altro utilizzo di questo tipo di tecniche è quello del tracciamento dei movimenti degli animali in natura, attraverso anche tecniche di triangolazione. Questa tecnica è usata sin dai primi anni 60 permettendo a ricercatori sul campo di stabilire la posizione di un animale ricevendo il segnale da diverse postazioni.
