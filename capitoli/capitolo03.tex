La situazione che inizialmente si è posta è stata quella di permettere ad un mezzo autonomo (ovvero in grado di operare senza azione esterna), ad esempio un robot dotato di ruote, o un multirotore, di raggiungere una postazione in grado di emettere un segnale. È stato quindi deciso di risolvere il problema primario, ovvero un algoritmo in grado di eseguire i passi necessari in un mondo simulato.




\section{Ricerca e informazione imperfetta}
L'agente che risolve il problema deve massimizzare la misura della prestazione, quindi per prima cosa è necessario fissare un obiettivo. Il compito dell'agente sarà quindi quello di trovare una sequenza di azioni che portano al raggiungimento di uno stato obiettivo. Fondamentale nel permettere di sviluppare una sequenza di scelte adeguate, è la fase di astrazione del problema, fatta per semplificare le condizioni in cui si è sviluppato l'algoritmo, prescindendo da problemi di carattere pratico, eliminando le problematiche di ingegnerizzazione. Un'astrazione è valida se possiamo espandere ogni soluzione astratta in una soluzione nel mondo dettagliato; l'utilità è quindi quella di rendere più facile l'esecuzione delle azioni nel mondo astratto rispetto a quello originale. 
La formulazione del problema è il processo che porta a decidere, dato un obiettivo, quali azioni e stati considerare. La considerazione degli stati è strettamente legata alle informazioni che l'agente dispone e che se non avesse non potrebbe fare altro che scegliere a caso. In generale un agente che ha a disposizione diverse opzioni immediate di valore sconosciuto, può decidere cosa fare esaminando diverse possibili sequenze di azioni che portano a stati di valore conosciuto, scegliendo quindi la sequenza migliore. Questo processo di selezione è detto ricerca, la cui implementazione ha come input un problema e restituisce la soluzione sotto forma di sequenza di azioni. La fase in cui l'agente svolge le azioni è detta esecuzione.
Nel problema analizzato l'agente dispone di informazioni limitate sullo stato del mondo, in quanto quest'ultimo non è completamente conosciuto dall'agente. È stato quindi necessario sviluppare una tecnica apposita, in quanto non si sarebbe potuto utilizzare un algoritmo classico di ricerca informata, quali l'A* o algoritmi greedy, perchè questi ultimi necessitano di informazione completa sugli stati del mondo, e non sarebbe stato conveniente utilizzare un algoritmo di ricerca completamente non informata, perchè non si sarebbe sfruttata l'informazione sugli stati in cui man mano l'agente si trova.


\section{Definizione delle componenti}
Le componenti che definiscono il mondo simulato sono:

-Mondo

-Agente

-Knowledge Base (KB)

-Gioco

\subsection{Mondo}
Il mondo è rappresentato sotto forma di matrice di dimensioni nxn, con un punto di coordinate (x, y) che rappresenta lo stato-obiettivo, in cui è memorizzato un valore numerico negativo, mentre in tutti gli altri punti viene memorizzato un valore float che indica la distanza euclidea del punto in cui il valore è memorizzato dal punto (x, y). Ad esempio nel punto (x+1, y) sarà memorizzato il valore "1.000", nel punto (X+1, Y+1) il valore "1.414", nel punto (X, Y) il valore "-1". In questo modo l'agente è in grado di sapere quando l'obiettivo è stato raggiunto, in quanto l'unico valore negativo nella matrice sarà quello del punto di arrivo.
\subsection{Agente}
L'agente è rappresentato da una classe, chiamata "Drone", che nel momento in cui viene istanziata inizializza le componenti necessarie a memorizzare le informazioni necessarie alla ricerca, il metodo che permette l'effettivo spostamente sul mondo, ed il metodo per l'acquisizione delle misurazioni provenienti dal mondo
\subsection{Knowledge Base (KB)}
Il KB sono le informazioni che il l'agente memorizza riguardo il mondo, ed è stato sviluppato sotto forma di grafo, che si va a riempire mano a mano che l'agente esegue i passi, che ha quindi un numero di nodi al massimo pari al numero di passi eseguiti per raggiungere l'obiettivo. Per ottimizzare l'utilizzo di memoria, è stata aggiunta la possibilità di limitare il numero di nodi memorizzati nel grafo ad un intero N, con una perdita in termini di prestazioni variabile, ma che permette un buon compromesso per determinati valori di N.
\subsection{Gioco}
Il gioco è la definizione della sequenza in cui svolgere le azioni per la simulazione: si inizializzano l'agente, il mondo ed il knowledge. Dopodichè, in un ciclo che finisce quando l'agente ha raggiunto l'obiettivo, o quando il valore "fuel" del Drone arriva a 0, si seleziona il Drone da un array, si chiama la funzione di probe del drone, per permettergli di misurare la propria distanza dal punto di arrivo, si assegnano a due variabili x ed y il rirultato dell'esecuzione dell'algoritmo di ricerca vero e proprio, e poi si chiama la funzione di movimento del drone con parametri queste due variabili. Alla fine del ciclo si aumento il contatore dei passi svolti, si controlla se il drone è arrivato a destinazione, se non lo è il ciclo ricomincia, altrimenti viene annunciato il successo.

	
\section{Architetture software}
Il software è strutturato in maniera modulare, in modo da permettere il riutilizzo delle singole componenti in diversi ambiti, o di mentenere ed aggiornare i moduli senza influenzare il funzionamento dell'intero programma, a patto di mantere la stessa interfaccia.
\subsection{Pacchetto "Game engine"}
Nel pacchetto Game Engine sono presenti i moduli che permettono l'utilizzo del sistema di simulazione
\subsubsection{Game.py}
Il modulo Game.py è la classe di gioco, che nel momento in cui ne viene creata un'istanza, inizializza le componenti del gioco (mondo, droni, knowledge base). All'interno è presente la funzione di gioco "start-game" in cui vengono eseguiti i passi di ogni turno di gioco. La selezione dell'agente viene fatta prelevando un oggetto Drone da un array di oggetti, per consentire la possibilità di sviluppare in seguito il gioco in maniera concorrente, facendo muovere più Droni nel mondo, e decretando vincitore il primo che raggiunge l'obiettivo. Questa funzionalità non era comunque richiesta ai fini della simulazione, quindi non è stata sviluppata. Subito dopo aver selezionato il Drone che deve fare la mossa, viene chiamata la funzione di rilevamento (probe) del valore della distanza, e quindi viene chiamata la "strategy" ovvero l'algoritmo che ritornerà il punto verso cui muoversi sotto forma di coordinate cartesiane.
\subsubsection{Drone.py}
Il modulo Drone.py è la classe che rappresenta l'agente. Quando viene istanziata, viene creato il grafo che rappresenta il mondo visitato dal Drone stesso, ed alcune variabili in cui vengono memorizzati dati che potranno essere utili per la scelta del prossimo passo. All'interno della classe troviamo la funzione "move" che svolge la funzione di diminuire di un'unita la variabile "fuel", aggiornare la posizione attuale e aggiungere il nodo della posizione al grafo. La funzione "strategy" serve semplicemente a richiamare l'algoritmo vero e proprio, e ritornare i valori x e y restituiti da quest'ultimo. La funzione "probe" salva il valore ricevuto dal "mondo" in un'array dinamico. E' inoltre presente una funzione "print-world" che se richiamata stampa a schermo l'attuale situazione del gioco, rappresentando i punti della matrice del mondo con un intero che indica il numero di volte che l'agente è passato su ogni casella.
\subsubsection{DroneKnowledge.py}
Inoltre è presente una classe DroneKnowledge che rappresenta un Drone con l'informazione completa sullo stato del mondo, utilizzato per eseguire il task attraverso un algoritmo greedy ottimo, che è stato usato per confrontare la qualità dell'algoritmo qui sviluppato.
\subsection{Pacchetto utils}
\subsubsection{Nodes.py}
Il modulo "nodes.py" contiene la struttura dati che viene utilizzata per rappresentare il mondo visitato dal Drone sotto forma di grafo. Ogni nodo del grafo è rappresentato da una tupla "k" che rappresenta le coordinate del punto, e viene utilizzata come chiave di ricerca dei nodi, una variabile intera "weight" che rappresenta il numero di volte in cui il drone è passato su un determinato nodo, e la variabile "counter", incrementata ad ogni operazione sul grafo, che serve ad indicare quando il Drone è passato l'ultima volta sul nodo (più alto è il valore, più di recente è stato utilizzato il nodo). 
La struttura a grafo è appunto composta di singoli nodi come sopra descritti. In fase di inizializzazione può essere impostata la variabile "graph-max-length" che permette di impostare un limite al numero massimo di nodi nel grafo, in caso di necessità di ottimizazione della memoria. L'interfaccia permette di prelevare il valore di un nodo date le coordinate, attraverso l'uso della funzione builtin "getitem" (ovvero si può accedere ai valori di un nodo con l'utilizzo della sintassi "grafo[(x, y)]" ). L'aggiunta di un nodo al grafo è gestita in modo da:

Aggiungere un nuovo nodo se le coordinate non sono ancora presenti all'interno del grafo;

Aggiornare un nodo se viene richiesto l'inserimento di coordinate già presenti, aumentando di 1 il valore weight, che indice il numero di volte in cui si è passato sul nodo, ed aggiornare il valore counter, in modo da memorizzare il fatto che questo è stato l'ultimo nodo usato

Nel caso in cui sia stato impostato un limite alla lunghezza del grafo, eliminare il nodo usato meno di recente prima di aggiungerne uno nuovo
\subsubsection{Matrix-generator.py}
Nel modulo "matrix-generator.py" sono presenti vari tool che permettono la generazione del mondo sotto forma di matrice utili per il sistema di gioco, mentre nel modulo "direction-modifier.py" sono presenti funzione che servono a modificare la direzione finale scelta dall'algoritmo nel caso questa superi i limiti della matrice, o in generale a gestire le varie eccezioni causate dalla dimensione finita della rappresentazione del mondo, onde evitare loop o errori. 
\subsection{Pacchetto Algorithm}
\subsubsection{Algorithm.py}
All'interno del pacchetto "algorithm" sono presenti gli algoritmi di ricerca, che verranno spiegati nel dettaglio in seguito. Questi ricevono in input due coordinate cartesiane e restituiscono nella stessa forma il risultato.


	
\section{Definizione delle strategie}
La strategia per la scelta della direzione in cui muovere il Drone è suddivisa in due fasi. Una calibrazione iniziale fatta per scegliere capire in che parte del piano rappresentante il mondo si trovi il punto di arrivo, e una strategia di movimento per avvicinarsi nel minor numero di passi al punto di arrivo. Al fine di migliorare l'efficenza ed evitare di incorrere in loop durante il percorso, in qualsiasi punto dell'algoritmo di ricerca viene fatto un controllo sul nodo in cui l'agente si trova, e nel caso sia passato per più di tre volte nello stesso punto questo si sposta in una delle caselle adiacenti in cui è passato il minor numero di volte e viene riavviato il processo di calibrazione.
La prima fase consiste in una "calibrazione" che serve a capire in quale parte del piano rispetto all'agente si trovi il punto di arrivo. L'agente si sposta quindi per prima cosa sul piano orizzontale, e comparando le due misurazioni è in grado di escludere la parte di piano a destra o a sinistra del punto di partenza, poi sul piano verticale, in modo da escludere la porzione di piano sopra o sotto il punto di partenza, poi inizia a muoversi nella direzione della bisettrice che taglia l'unica porzione di piano dove il punto di arrivo può trovarsi. Dopo un passo sulla bisettrice, viene valutato di quanto ci si è avvicinati al punto di arrivo. Se il punto è sulla bisettrice, la differenza fra l'ultima distanza misurata e la precedente dovrà essere di circa 1.414, se questo valore è minore, significa che il punto si può trovare sopra o sotto alla bisettrice. A questo punto viene eseguita l'unica scelta "random" dell'algoritmo, ovvero essendo uguale la probabilità che il punto di arrivo sia sopra o sotto la bisettrice, l'agente si sposta in direzione verticale od orizzontale. Se con quest'ultimo spostamento si è allontanato dalla sorgente, l'agente si muove in direzione perpendicolare rispetto all'ultimo movimento, tornando sulla bisettrice, altrimenti viene svolta ancora una valutazione: se la differenza fra l'ultima distanza e quella precedente è pari a 1, significa che il punto di arrivo è sulla retta in cui si è appena mosso, e il Drone continuerà in quella direzione finco al raggiungimento dell'obiettivo, altrimenti si muoverà in direzione diagonale, parallelo rispetto alla bisettrice, e tutto il processo viene iterato. Questa fase della ricerca si interrompe in due casi: 

Nel caso in cui, durante uno dei movimenti in diagonale, il Drone si stia allontando dalla sorgente, viene rieseguita la calibrazione e il processo rinizia da capo

Nel caso in cui il drone arrivi a destinazione
	
\section{Esecuzione dell'algoritmo}
	guarda qua