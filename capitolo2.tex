\chapter{Soluzione del problema con un algoritmo di posizionamento ad ancora singola}
\fancyhead[RO]{\bfseries Soluzione del problema con un algoritmo di posizionamento ad ancora singola}
\section{Problemi di posizionamento}
Il problema del posizionamento in generale consiste nello stabilire il punto geografico in cui si trova un oggetto. In ambienti aperti, e per le lunghe distanze, in generale questo viene svolto da sistemi GPS, che si sono ampiamente diffusi negli ultimi anni grazie anche alla possibilità di essere inclusi negli smartphone a costi contenuti. Questa tecnologia però non funziona  ad esempio all'interno di edifici o perde di precisione in condizioni atmosferiche avverse, e quindi in alcuni casi, per posizionamento a medio-corto raggio, è ancora necessario utilizzare tecnologie differenti. In questi casi è comune l'utilizzo di un certo numero di ancore, posizionate nello spazio dove deve essere effettuato il posizionamento, utilizzate come punti di riferimento per effettuare la ricerca. Per permettere alle ancore di comunicare fra di loro e con la ricevente si utilizzano tecnologie wireless. Si cerca quindi di stabilire la posizione relativa di una serie di "ancore" (che trasmettono un segnale) rispetto ad una ricevente che si trova nall'interno dello spazio coperto dal segnale delle stesse.
A questo fine sono utilizzati algoritmi e sistemi basati su reti di sensori che comunicano uno con l'altro scambiandosi informazioni riguardo l'ambiente in cui si trovano. Grazie a queste è possibile stabilire la posizione di ogni nodo rispetto agli altri, e raccogliere dati su un ambiente al fine di descriverlo. 
Questi sono sistemi di radio positioning, che utilizzano le tecniche precedentemente descritte, quali il DF basato sul tempo di arrivo (TOA) del segnale o misurazioni di fase o misurazioni RSS. Ognuna di queste tecniche può essere usata allo scopo, la differenza è nel tipo di tecnologie usate, nel costo dell'attrezzatura necessaria, nella richiesta di energia e ovviamente nell'accuratezza del risultato finale.
In genere è richiesto l'utilizzo di almeno 4 nodi in uno spazio tridimensionale, per avere un'accuratezza sufficente del posizionamento di un oggetto in una posizione non nota dello spazio di riferimento. Verrà qui introdotto un algoritmo che permette questo tipo di localizzazione in una simulazione in 2 dimensioni del mondo, utilizzando un solo nodo base, e una ricevente mobile. Questo comporta dei vantaggi dal punto di vista di costo ed energia necessaria, riducendo il numero di componenti necessari, utilizzando tecnologie presenti in ogni dispositivo in grado di comunicare via radio (wifi, bluetooth, e in generale qualsiasi apparecchio da cui sia possibile ricavare la potenza del segnale ricevuto) ed utilizzando quindi un algoritmo relativamente leggero.
	
\section{Risoluzione di problemi di direction finding con il metodo di posizionamento ad ancora singola}
L'idea  qui sviluppata è quella di permettere di determinare la posizione di una singola ancora sorgente, attraverso una ricevente mobile che effettui misurazioni ad ogni passo svolto nello spazio. La misurazione utilizzata sarà quella della potenza di segnale (RSS), che verrà utilizzata, attraverso una scala del tipo quella presentata nel primo capitolo, per determinare la distanza dalla sorgente, in campo aperto.
La principale differenza con gli approcci comunemente usati sta nell'utilizzo di una singola base trasmettitrice. 
Il vantaggio di questo approccio è sicuramente il basso costo dei componenti, in quanto è necessario una singola ricevente e un singolo trasmettitore, senza bisogno di sensori o costosi apprecchi in grado di rilevare il tempo di ricezione con un errore di pochi milionesimi di secondo.

Per trovare una soluzione al problema descritto, è stato deciso di lavorare prima di tutto all'interno di una simulazione ideale del mondo, in cui si trascurano gli errori di rilevazione, la presenza di interferenze al segnale, problematiche strettamente correlate allo spostamento della ricevente stessa, e in generale tutte le problematiche che verrebbero affrontate in fase di sviluppo nel mondo reale. Questo per permettere di concentrarsi anzitutto sull'aspetto principale del problema: la scrittura di un algoritmo che permetta di localizzare e raggiungere una sorgente radio attraverso misurazioni di tipo RSS. È necessario stabilire anche un valore che permetta di stabilire la bontà dell'algoritmo stesso. In questo caso l'indice di qualità del processo di localizzazione sarà il numero di spostamenti effettuati dall'agente prima di raggiungere la posizione della sorgente.

Il problema da affrontare è quindi \textbf{la scelta della direzione in cui far muovere la ricevente per permettere di raggiungere la posizione della sorgente nel minor numero possibile di passi}. Il problema verrà risolto utilizzando un approccio geometrico. Il mondo viene rappresentato come un griglia di "caselle" in cui la ricevente potrà muoversi di un passo alla volta. La rilevazione della potenza del segnale diviene quindi una proprietà di ogni casella (in quanto non varia durante l'esecuzione, poichè la sorgente è fissa in un punto) e verrà associata ad ognuna di queste. Tale valore sarà la distanza euclidea del punto della griglia dal punto in cui si trova la sorgente. La ricevente potrà muoversi in ognuna delle caselle adiacenti, quindi sono ammessi anche spostamenti in diagonale. Grazie al fatto che la simulazione è stata sviluppata in maniera modulare, sarà possibile confrontare l'algoritmo sviluppato con altri algoritmi, utilizzare l'algoritmo stesso in ambiti diversi, purchè venga mantenuta la stessa interfaccia, eseguire test che forniscono dati statistici utili a valutare la bontà dell'algoritmo qui sviluppato, e di qualsiasi altro algoritmo che può essere sviluppato per risolvere lo stesso tipo di problema. 


