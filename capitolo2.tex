\chapter{Soluzione del problema con un algoritmo di posizionamento ad ancora singola}
\fancyhead[RO]{\bfseries Soluzione del problema con un algoritmo di posizionamento ad ancora singola}
\section{Problemi di posizionamento}
Il problema del posizionamento consiste nello stabilire il punto geografico in cui si trova un oggetto. In ambienti aperti, e per le lunghe distanze, vengono spesso utilizzati sistemi GPS, la cui evoluzione ha reso possibile la loro integrazione negli smartphone e nei dispositivi embedded a costi contenuti. Questa tecnologia ha dei limiti in quanto, ad esempio, non funziona all'interno di edifici e perde di precisione in condizioni atmosferiche avverse. Per questo, problemi di posizionamento a medio-corto raggio richiedono tutt'ora l'impiego di tecnologie  e metodi differenti. Un possibile approccio prevede l'utilizzo di "ancore". Un'ancora è un dispositivo in gradi di raccogliere dati attraverso dei sensori, e di trasmetterli ad una ricevente. La ricevente memorizza i dati raccolti dalle ancore ed effettua i calcoli necessari. La comunicazione tra le ancore e la ricevente può essere implementata tramite onde radio. Si cerca quindi di stabilire la posizione relativa di una serie di ancore rispetto ad una ricevente che si trova nall'interno dello spazio coperto dal segnale delle stesse.
A questo fine sono utilizzati algoritmi e sistemi basati su reti di sensori che comunicano con un nodo centrale (o anche uno con l'altro) scambiandosi informazioni riguardo l'ambiente in cui si trovano. Grazie a questi è possibile stabilire la posizione di ogni ancora rispetto alle altre, e raccogliere dati su un ambiente al fine di descriverlo. 
Questi sono sistemi di radio positioning che utilizzano le tecniche precedentemente descritte, quali il DF basato sul tempo di arrivo del segnale, la misurazione di fase o la misurazione RSS. Ognuna di queste tecniche può essere usata a tale scopo e la differenza sta nel tipo di tecnologie utilizzate, nel costo dell'attrezzatura necessaria, nella richiesta energetica e nell'accuratezza del risultato finale.
In generale sono previsti almeno 4 nodi in uno spazio tridimensionale, così da avere una buona accuratezza nel rilevamento di un punto in uno spazio non noto. In questo lavoro sarà introdotto un algoritmo geometrico che permette questo tipo di localizzazione in una simulazione a 2 dimensioni, con una sola ancora e una ricevente mobile. L'utilizzo di una sola ancora comporta dei vantaggi dal punto di vista di costo ed energia richiesta. Si riduce quindi il numero di componenti necessari, utilizzando tecnologie presenti in ogni dispositivo in grado di comunicare via radio ed utilizzando quindi un algoritmo relativamente leggero.
	
\section{Risoluzione di problemi di direction finding con il metodo di posizionamento ad ancora singola}
L'idea è quella di determinare la posizione di una singola ancora sorgente attraverso una ricevente mobile che effettui misurazioni ad ogni passo percorso nello spazio. La tecnica utilizzata sarà la misurazione della potenza di segnale (RSS). Questa verrà convertita in distanza utilizzando una scala di conversione come quella presentata nel primo capitolo.
La principale differenza con gli approcci comunemente usati sta nell'utilizzo di una singola ancora. 
Il vantaggio di questo approccio è sicuramente il basso costo dei componenti, in quanto è necessario una singola ricevente e un singolo trasmettitore, senza la necessità di sensori o costosi apprecchi in grado di rilevare il tempo di ricezione con un errore di pochi microsecondi.

Per trovare un algoritmo di risoluzione del problema descritto si è scelto di operare all'interno di una simulazione ideale del mondo in cui si trascurano gli errori di rilevazione, la presenza di interferenze del segnale, problematiche strettamente correlate alla meccanica di spostamento ed a tutte le problematiche che sarebbero affrontate in fase d'implementazione reale. Per stabilire la bontà dell'algoritmo stesso è stato preso come indice il numero di spostamenti effettuati dall'agente prima di raggiungere la posizione della sorgente.

Il problema da affrontare è quindi \textbf{la scelta della direzione in cui far muovere la ricevente per permettere di raggiungere la posizione della sorgente nel minor numero possibile di passi}. Il problema sarà risolto utilizzando un approccio geometrico. Il mondo sarà rappresentato come una matrice in cui la ricevente potrà muoversi di un passo alla volta. La rilevazione della potenza del segnale diviene quindi il valore di ogni elemento, che non varierà durante l'esecuzione dell'algoritmo poichè la sorgente resterà fissa in un punto dello spazio. Tale valore rappresenta la distanza euclidea tra l'elemento della matrice ed il punto in cui si trova la sorgente. La ricevente potrà muoversi in ognuna degli elementi adiacenti nella matrice, consentendo anche spostamenti in diagonale. Poiché la simulazione è stata sviluppata in maniera modulare, è possibile confrontare l'algoritmo sviluppato con altri algoritmi. È anche possibile utilizzare l'algoritmo stesso in ambiti diversi, purchè venga mantenuta la stessa interfaccia di programmazione. Inoltre la suite di test fornisce dati statistici utili a valutare la bontà di un qualsiasi algoritmo sviluppato per risolvere lo stesso tipo di problema. 
