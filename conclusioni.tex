\chapter*{Conclusioni}
In questa tesi ci siamo occupati della realizzazione di un software che simula un problema reale, quello del direction finding. Alla base dello sviluppo c'è il linguaggio di programmazione Python.

L'intera simulazione è stata sviluppata con la struttura di un gioco a turni.
È stato sviluppato un sistema di memorizzazione dei dati dinamico, che permette di ottimizzare l'utilizzo della memoria.

L'evoluzione del codice, e lo sviluppo di questa stessa tesi, sono visionabili grazie al sistema di versionamento GIT utilizzato durante tutto lo sviluppo.

Come ultimo punto, ma non meno importante, è giusto notare che tutte le tecnologie software utilizzate sono completamente Open Source, il che rende il progetto completamente libero e accessibile per futuri miglioramenti.