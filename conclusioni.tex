\chapter*{Conclusioni}
Osservano i risultati ottenuti, l'algoritmo si comporta bene nella maggior parte dei casi, in media compie il doppio dei movimenti di un algoritmo ottimo che possiede tutte le informazioni disponibili sul mondo, stesso comportamento anche per il caso peggiore. 

Lo scarto quadratico basso è indice del fatto che in genere i risultati sono vicini alla media aritmetica, questo significa che l'algoritmo si comporta in maniera coerente. 

Probabilmente è possibile migliorare ancora le prestazioni dell'algoritmo analizzando le configurazioni in cui il risultato è peggiore. 

Il progetto in sè, grazie alla struttura del software, e alla presenza di una suite di benchmark, è comunque predisposto ad essere riutilizzato.