\chapter*{Introduzione}
Lo sviluppo e la costante diffusione di tecnologie di comunicazione wireless aprono nuove frontiere alla progettazione ed alla realizzazione di dispositivi connessi fra di loro. Basti pensare ai sistemi di geolocalizzazione presenti negli smartphone che permettono di avere informazioni riferite al luogo in cui ci si trova in tempo reale, o ai navigatori gps, già in largo uso da diversi anni. Questa evoluzione è andata di pari passo con nuove tecniche di ""context awereness'' che hanno permesso ai dispositivi di riconoscere il tipo di ambiente in cui si trovano, attraverso l'analisi di dati provenienti da sensori così da reagire di conseguenza. 

Ad esempio uno smartphone ""cosciente'' del contesto in cui si trova, potrebbe riconoscere una stanza dove è in corso una riunione e rifiutare una chiamata in arrivo. Questa funzionalità può essere estesa anche in ambito medico, dove un dispositivo può visualizzare a schermo i dati del paziente, oppure presentare i compiti da svolgere in una determinata zona dell'ospedale così da permettere di inserire o aggiornare questi dati. 

Un altro possibile scenario, visto il crescente interesse nei confronti di questi mezzi, riguarda i droni\footnote{aeromobile multirotore in grado di muoversi autonomamente, senza bisogno di controlli dall'esterno}. Questi possono essere utilizzati per raggiungere parti del mondo in cui le strade non sono utilizzabili se non in brevi periodi dell'anno, per portare cibo, medicinali e beni di prima necessità. Come propone Andreas Raptopoulos durante la conferenza TED Global di giugno 2013\footnote{ \url{http://www.ted.com/talks/andreas_raptopoulos_no_roads_there_s_a_drone_for_that} }, sviluppare un network di basi di atterraggio e ricarica per droni che copra una vasta superficie geografica è molto più economico, veloce, ed ecocompatibile della costruzione e manutenzione di arterie stradali o ferroviarie.

In questo lavoro di tesi è stato preso in considerazione quest'ultimo problema. È stato sviluppato un software che simula questa situazione utilizzando il linguaggio di programmazione Python, già utilizzato in altri progetti che sviluppano tecniche di ""context awareness''.

È stato creato un sistema di gioco all'interno del quale fosse possibile sviluppare l'algoritmo di controllo del drone che permettesse di calcolare il percorso da fare per raggiungere un punto nello spazio. Per analizzarne il comportamento è stato sviluppato un software di benchmark che permettesse di valutarne la bontà in maniera sistematica, simulando quante più configurazioni possibili.

Durante tutto questo processo è stato svolto un lavoro di ricerca che ha permesso di inquadrare la tipologia del problema.

Nel primo capitolo sarà introdotto il concetto di ""direction finding'' e saranno spiegate le diverse tecniche impiegate per risolvere il problema. 

Nel secondo capitolo si affronta il problema dal punto di vista del ""positioning'', ovvero della localizzazione di un punto in uno spazio geografico. 

Nel terzo capitolo viene analizzato tutto il lavoro svolto nella fase di implementazione, evidenziandone le parti più importanti ed entrando anche nel dettaglio delle varie componenti create.

Il quarto capitolo contiene i risultati della suite di benchmark, e le conseguenti riflessioni.