\chapter*{Introduzione}
Lo sviluppo e la costante diffusione di tecnologie di comunicazione wireless aprono nuove frontiere per la progettazione e la realizzazione di dispositivi in grado di sfruttare questa potenzialità. Basti pensare ai sistemi di geolocalizzazione presenti negli smartphone che permettono di avere informazioni riferite al luogo in cui ci si trova in tempo reale, o ai navigatori gps, già in largo uso da diversi anni. Questa evoluzione è andata di pari passo all'evoluzione di tecniche di "context awereness" ovvero alla capacità dei dispositivi di riconoscere il tipo di ambiente in cui si trovano, attraverso l'analisi di dati provenienti da sensori per poi comportarsi di conseguenza. 

Ad esempio uno smartphone che sia cosciente del contesto in cui si trova, potrebbe sapere di trovarsi in una stanza dove è in corso una riunione e rifiutare una chiamata in arrivo. In ambito medico un dispositivo può visualizzare a schermo i dati del paziente di fronte a cui ci si trova, presentare i compiti da svolgere in una determinata zona dell'ospedale e permettere di inserire o aggiornare questi dati. 

Altro possibile sviluppo, visto il crescente interessi nei confronti di questi mezzi, è permettere ad un drone\footnote{aeromobile multirotore in grado di muoversi autonomamente, senza bisogno di controlli dall'esterno} di decollare da un punto e raggiungere un dispositivo bluetooth messo ad una certa distanza, atterrando nelle sue vicinanze. 

Nel nostro lavoro di tesi si è riprodotto proprio questo problema, attraverso un software che simula questa situazione. È stato scelto di utilizzare il linguaggio di programmazione Python, già utilizzato in altri progetti che sviluppano tecniche di context-awareness.

È stato quindi creato un sistema di gioco, all'interno del quale fosse possibile sviluppare l'algoritmo, e studiarne il comportamento nelle diverse situazioni. 

Per analizzare il comportamento dell'algoritmo in ogni possibile situazione è stato quindi sviluppato un software di benchmark che permettesse di valutare la bontà dell'algoritmo in maniera sistematica.

Durante tutto questo processo è stato svolto un lavoro di ricerca che ha permesso di inquadrare la tipologia del problema.

Nel primo capitolo verrà quindi introdotto il concetto di direction finding, e vengono spiegati i diversi utilizzi e tecniche impiegati. 

Nel secondo capitolo si affronta il problema dal punto di vista del "positioning", ovvero la localizzazione di un punto in uno spazio geografico. 
Nel terzo capitolo viene analizzato invece tutto il lavoro svolto nella fase di scrittura del codice, evidenziandone le parti più importanti, entrando anche nel dettaglio dell'implementazione. 

Il quarto capitolo contiene i risultati avuti dalla suite di benchmark, e le riflessioni riguardo i risultati ottenuti.