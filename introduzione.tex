\chapter*{Introduzione}
Lo sviluppo e la sempre maggiore diffusione di tecnologie di comunicazione wireless sta aprendo ogni giorno nuove frontiere nell'ideazione di apparecchi in grado di sfruttare questo network in maniere innovative. Basti pensare ai sistemi di geolocalizzazione negli smartphone, che permettono di avere informazioni in tempo reale riferite al luogo in cui ci si trova, senza il bisogno che questo venga esplicitamente specificato, ai navigatori gps, ed in generale qualsiasi tipo di tecnologia che richieda un interazione con l'ambiente circostante. Tutto questo è possibile grazie a tecniche di context awereness, ovvero alla capacità dei dispositivi di riconoscere in qualche modo il tipo di ambiente in cui si trovano, attraverso l'analisi di dati provenienti da sensori o altro, e di comportarsi di conseguenza. Ed è da qui che inizia il percorso che mi ha portato alla stesura di questa tesi. Mi è stato proposto, come obiettivo del tirocinio, di cimentarmi nello sviluppo di una tecnica di context awareness. Si è quindi iniziato a pensare a delle possibili idee pratiche verso cui indirizzare lo sviluppo, e quella che è poi risultata più interessante è la seguente: permettere ad un drone (un aeromobile multirotore in grado di muoversi autonomamente, senza bisogno di controlli dall'esterno) di decollare da un punto e raggiungere un qualunque dispositivo bluetooth messo ad una certa distanza, atterrando nelle sue vicinanze. Il problema così presentato nascondeva diverse complicazioni. È stato quindi necessario procedere per passi, dimenticare per un momento l'idea di vedere subito il drone volare, e cercare invece un metodo di lavoro che potesse permettere di gettare solide basi per eventuali sviluppi futuri. Quindi il primo ostacolo da risolvere era quello di analizzare il problema, e capirne gli elementi fondamentali, per evitare di concentrarsi su aspetti secondari. E in questa fase sono state utili le ore spese a muoversi intorno ad una stanza con un cellulare in mano controllando dove arrivasse il segnale bluetooth, quantomeno per capire che l'approccio diretto non era la via giusta! A quel punto era chiara quale fosse la strada corretta: cercare di riprodurre il problema con un software che potesse "muoversi intorno alla stanza" al mio posto. E da qui è iniziato lo sviluppo del progetto che viene presentato in questo testo. Ho scelto di utilizzare il linguaggio di programmazione Python, un po' perchè è stato uno dei primi che abbia usato, e con il quale ho scoperto il mondo della programmazione, un po' perchè nell'azienda dove ho svolto il tirocinio c'era un buon expertise a riguardo, un po' perchè rispetto ad altri linguaggi permette di scrivere codice in maniera più veloce. Una volta sviluppato il sistema di gioco, ed aver speso diverso tempo nell'adattamento dell'algoritmo alle diverse situazioni che si presentavano durante lo sviluppo, è stato necessario analizzare tutto il lavoro svolto per poter presentare il software come tesi conclusiva di questa laurea triennale. È quindi seguito un lavoro di ricerca e studio che mi ha portato ad inquadrare quale fosse la tipologia del problema (direction finding). In seguito è sembrato opportuno poter misurare la qualità del lavoro fatto, quindi è seguito un aggiornamento del codice per permettere di utilizzare altri algoritmi senza modificare l'ambiente, cosa che è stata possibile grazie allo sviluppo modulare intrapreso fin dall'inizio della scrittura del codice. Una volta completato il sistema di gioco, per poter effettuare dei veri test sul software, si è rivelato utile scrivere una suite di test che permettesse di effettuare le diverse prove, ne raccogliesse i risultati, e, dopo averli analizzati, potesse permettere di paragonare risultati ottenuti da algoritmo di ricerca differenti per valutarne la bontà in termini di prestazioni.
Nel primo capitolo verrà quindi introdotto il concetto di direction finding, e vengono spiegati i diversi utilizzi e tecniche impiegati. Nel secondo capitolo si affronta il problema dal punto di vista del "positioning", ovvero la localizzazione di un punto in uno spazio geografico. Nel terzo capitolo viene analizzato invece tutto il lavoro svolto nella prima fase, scomponendo il codice ed evidenziando le parti più importanti, entrando anche nel dettaglio dell'implementazione. Il quarto capitolo contiene i risultati avuti dalla suite di benchmark, e le riflessioni riguardo i risultati ottenuti.